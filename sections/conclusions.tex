\begin{ParaColumn}[\bisection*{Conclusions}{结论}]
    
    The compilation of large drained triaxial tests on coarse rockfill materials presented in this article includes samples with different relative densities, mineralogies, gradings, and particle shapes, resulting in significant data scatter. However, the minimum limit imposed on $d_{\max}$ reduces the potential for size effects, allowing an appropriate comparison with data on ballast and quartzitic dense sands, which are usually taken as a reference for mechanical parameters of granular materials.

    \switchcolumn

    本文介绍的对粗碎石材料进行的大排水三轴试验的汇编包括具有相对密度,矿物成分,级配等级和颗粒形状不同的样品,从而导致大量数据分散。 但是,对$d_{\max}$施加的最小的限制减小了尺寸影响的可能性,从而可以与石渣和石英质致密砂土的数据进行适当的比较,这些数据通常被用作颗粒材料力学参数的参考。
    
    \switchcolumn*

    The analyses show the typical behavior of crushable granular materials previously reported, namely, shear strength decreasing due to particle crushing when pressure increases. In this study, the magnitude of this phenomenon is highlighted in coarse, angular materials. In general, data on coarse rockfill materials confirm previously reported limits for average shear strength. Dense quartzitic sands reasonably correspond to the range of average to low shear strength values for rockfills in a large range from low to high pressure. At low pressure of $\sigma_n^\prime<0.2$ MPa, where the amount of crushing is still not significant, rockfills and ballasts have consistently higher shear strength than dense quartzitic sands, and the maximum internal friction angle is higher than 45°. At high pressure, the amount of grain breakage increases and the maximum internal friction angle of sands and rockfills is between 30° and 40°. According to the analysis, previous limits for low shear strength proposed in the literature seem conservative at low stresses. A new expression is provided in this paper that considers higher strength at normal stress lower than 0.7 MPa.

    \switchcolumn

    分析显示了先前报道的可破碎粒状材料的典型行为,即当压力增加时,由于颗粒破碎,抗剪强度降低。在本研究中,这种现象的量级在粗粒、有棱角的材料中得到了突出。一般来说,粗粒材料的数据证实了先前报告的平均抗剪强度极限。致密石英岩砂合理地对应于低压力到高压力大范围内的平均到低抗剪强度的范围。在低压$\sigma_n^\prime<0.2$MPa时,破碎量仍然不显著的情况下,碎石和石渣的抗剪强度始终高于致密石英砂,最大内摩擦角大于45°。在高压下,颗粒破碎量增大,砂土与碎石的最大内摩擦角在30°到40°之间。根据分析,以往文献中提出的低抗剪强度限值在低应力下显得保守。本文提出了一种考虑高强度时法向应力小于0.7MPa的新表达式。

    \switchcolumn*

    In general, dense quartzitic sands are stiffer than rockfills at low stresses and their secant Young’s moduli tends to have similar values at high stresses, probably due to the significant amount of particle crushing in both materials. Normalized Young’s moduli at reference pressures for rockfills are in the same typical range for loose sands.

    \switchcolumn

    通常,致密的石英砂在低应力下比碎石坚硬,其割线杨氏模量在高应力下往往具有相似的值,这可能是由于两种材料中大量的颗粒被压碎造成的。 对于松散砂体,参考压力下的归一化杨氏模量在相同的典型范围内。

\end{ParaColumn}